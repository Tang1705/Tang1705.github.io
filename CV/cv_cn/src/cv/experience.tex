%-------------------------------------------------------------------------------
%	SECTION TITLE
%-------------------------------------------------------------------------------
\cvsection{项目设计}


%-------------------------------------------------------------------------------
%	CONTENT
%-------------------------------------------------------------------------------
\begin{cventries}

	\cventry
	{项目负责人} % Job title
	{联合单目深度估计的深度图像超分辨率重建算法研究 \href{https://dl.acm.org/doi/pdf/10.1145/3474085.3475373}{{\color{awesome}\faChain}}
	%\href{http://tang5618.com/data/pdf/wordpress/tq.pdf}{{\color{awesome}\faFilePdfO}}
	} % Organization
	{北京交通大学信息科学研究所\newline\vspace{-0.5mm}\hspace{-10mm} 现代信息科学与网络技术北京市重点实验室} % Location
	{2020年11月--2021年6月} % Date(s)
	{
		\begin{cvitems} % Description(s) of tasks/responsibilities
			\item {\textbf{项目简介:}探索深度图超分辨率重建任务中颜色引导、细节恢复、模态交互等问题的解决方案,从多任务学习的角度出发研究一种联合深度估计的深度图超分辨率网络,并探索两个任务之间的交互指导关系,以达到相互促进、互利共赢的效果。}
			\item {\textbf{项目成果:}以第一作者撰写的论文被多媒体领域顶级会议录用(ACM International Conference on Multimedia,CCF A),发明专利申请初步审查合格(一种联合单目深度估计的深度图像超分辨率重建方法,申请号: 202110803976.2),北京市本科生优秀毕业论文(设计) \href{https://github.com/rmcong/BridgeNet_ACM-MM-2021}{{\color{awesome}\faGithub}}}。
			\item {\textbf{主要工作:}负责模型的设计、实现,论文的撰写、投稿以及专利申请技术交底书的撰写。}
			\item {\textbf{开发工具:}Python、PyTorch、MindSpore}
		\end{cvitems}
	}
	
	%---------------------------------------------------------
	\cventry
	{项目负责人} % Job title
	{基于编码结构光的高铁轮轨姿态三维重建 \href{http://gjcxcy.bjtu.edu.cn/NewLXItemListForStudentDetail.aspx?ItemNo=594113&year=2020&type=student&IsLXItem=1}{{\color{awesome}\faChain}}
	%\href{http://tang5618.com/data/pdf/wordpress/tq.pdf}{{\color{awesome}\faFilePdfO}}
	} % Organization
	{北京交通大学计算机与信息技术学院\newline\vspace{-0.5mm} 轨道交通智能检测与监测研究所} % Location
	{2019年4月--2020年7月} % Date(s)
	{
		\begin{cvitems} % Description(s) of tasks/responsibilities
			\item {\textbf{项目简介:}高铁轮轨姿态反映了车轮与钢轨之间复杂的动态相互作用和约束关系,获取高精度高铁轮轨姿态对于保障高速铁路安全运营具有重要的意义。本项目以机器视觉理论和方法为基础,重点研究基于编码结构光的高铁轮轨姿态三维重建方法。通过编码结构光获取轮轨稠密三维点云数据,三维重建高铁轮轨姿态模型,并实现可视化。}
			\item {\textbf{项目成果:}国家级大学生创新训练计划项目,项目采用基于空间编码的编码结构光的方法,将单幅编码图案投影在轮轨表面,提高特征点的提取和识别精度,并将 De Bruijn 分析与小波变换分析相结合,增加了基于特征点的点云提取稠密度,从而实现了单次投影的点云稠密重建。项目完成了从半径95mm的球体提取17W条以上的点云数据,半径误差0.678mm,实现了对铁轨等多个物体的三维重建,完成了结构光三维重建软件开发,提供了基于主动视觉的三维重建和点云数据可视化展示、编辑的平台,并入选 Github 排名前 99 的 C++ 三维重建开源项目(Awesome Open Source) \href{https://github.com/Tang1705/Reconstruction}{{\color{awesome}\faGithub}}}。
			\item {\textbf{主要工作:}项目负责人,负责项目管理,论文的学习和复现,结构光编码图案和解码算法的设计与实现以及软件的设计和开发。}
			\item {\textbf{开发工具:}C++、OpenCV、PCL、QT}
		\end{cvitems}
	}
	
	
	
	%---------------------------------------------------------
	\cventry
	{计算机视觉部分开发} % Job title
	{基于计算机视觉的智慧养老系统 \href{https://tang5618.com/wordpress/?p=1053}{{\color{awesome}\faChain}}} % Organization
	{北京交通大学软件学院} % Location
	{2020年6月--2020年7月} % Date(s)
	{
		\begin{cvitems} % Description(s) of tasks/responsibilities
			\item {\textbf{项目简介:}系统利用计算机视觉技术,对多组摄像头实时拍摄的画面中老人的微笑、摔倒、和义工互动等活动进行检测。系统由 Web 界面和摄像头两部分组成。管理员使用系统管理老人、义工,查看实时监控,得到实时的事件报警,如陌生人入侵和追踪等。}
			\item {\textbf{项目成果:}综合专题实践项目,系统借助 FaceNet 实现了单样本人脸验证和陌生人检测;通过基于 fer2013 数据集预训练的 Mini-Xception 对老人进行实时微笑检测;基于 OpenPose 的人体关键点和 Background Subtraction 提取人体运动特征从而对画面中的老人进行摔倒检测;通过伽罗华域下的伪随机矩阵编码和解码对系统标定,从而进行老人与义工互动检测等 \href{https://github.com/Tang1705/CVofSSE}{{\color{awesome}\faGithub}}。}
			\item {\textbf{主要工作:}项目负责人,负责项目管理和系统计算机视觉任务的需求分析、模型设计与实现。}
			\item {\textbf{开发工具:}Python、Django、SQL、Vue、Nuxt.js}
		\end{cvitems}
	}
	%---------------------------------------------------------
	%---------------------------------------------------------
%	\cventry
%	{用户界面设计与实现} % Job title
%	{基于 cocos 的宝石迷阵游戏 \href{http://tang5618.com/wordpress/?p=571}{{\color{awesome}\faChain}}} % Organization
%	{北京交通大学软件学院} % Location
%	{2019年5月--2019年6月} % Date(s)
%	{
%		\begin{cvitems} % Description(s) of tasks/responsibilities
%			\item {\textbf{项目简介:}宝石迷阵是一款交换消除游戏,游戏画面会出现各式各样的宝石,基本规则是通过交换相邻的宝石,使3个同一颜色的宝石连在一起,即可消去他们。本项目选择 cocos 游戏引擎,在经典游戏的基础上,对游戏风格和规则进行设计和实现,提供不同的主题以及经典模式和进阶模式两大模式。模式中分别有练习、闯关和挑战三种小模式。练习模式方便用户练习,闯关模式共设有 10 个关卡,难度递增,通关后即可进入挑战模式,在有限步数内挑战尽可能高的分数,最终成绩会上传到排行榜。除各类游戏功能外,游戏还提供在社交平台分享成绩等辅助功能。}
%			\item {\textbf{项目成果:}数据结构课程设计,提供了 win32 的 Release 版本,GitHub Commit 64/318次 \href{https://github.com/Tang1705/IconBattle}{{\color{awesome}\faGithub}}}。
%			\item {\textbf{主要工作:}负责游戏界面(包括布局、动画等)的设计和实现以及代码质量审查;}
%			\item {\textbf{开发工具:}C++、cocos}
%		\end{cvitems}
%	}
\end{cventries}
